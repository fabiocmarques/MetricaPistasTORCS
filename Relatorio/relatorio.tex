\documentclass{article}

\usepackage[utf8]{inputenc}
\usepackage[portuguese]{babel}
\usepackage{natbib}

\textheight = 650pt
\textwidth = 350pt
\marginparwidth = 5pt

\title{Esboço do Relatório}
\author{Fábio Costa Farias Marques, 14/0039082}
\date{28/03/2016}

\begin{document}

\maketitle
\section{Introdução}
	\begin{itemize}
		\item \textbf{Parte inicial}: apresentação do problema e da proposta, citando também os méritos do projeto, sendo elas:  aplicação na área de geração procedural de conteúdo, de modo a criar pistas de acordo com uma dificuldade pré-definida por uma métrica;
		\item \textbf{TORCS}\cite{TORCS}: apresentação e breve análise de algumas características do jogo;
		\item \textbf{SCRC}: apresentação do campeonato e colocá-lo motivador do projeto, assim como descrever uma futura implementação para esta versão do TORCS, seja para avaliação das pistas ou até mesmo de análise de desempenho dos pilotos; 
	\end{itemize}

\section{Metodologia}
	\begin{itemize}
		\item \textbf{Pistas}: As pistas do TORCS\cite{TORCS} ;

		\item \textbf{Leitura de XML}: Uma ferramenta na linguagem \textit{Java}\cite{Java} foi desenvolvida para a leitura dos arquivos ``.xml'' das pistas, fazendo uso da API popularmente conhecida como \textit{SAX}\cite{SAX}, disponível em \cite{meuGit}. As pistas e cada um dos segmentos foram modelados como objetos, sendo que este foi estendido para seus dois tipos básicos: curvas, com seus atributos de nome, arco, raio e para qual lado ela leva, tendo como referência de posição seu início; e as retas, sendo compostas somente por seu nome e comprimento. As pistas são compostas somente por um objeto auxiliar, usado na aquisição das informações do arquivo, e uma lista ordenada de seus segmentos, de acordo com sua disposição no ``.xml''. Devido à heterogeneidade das definições das curvas, nas quais não existe um padrão de utilização do campo ``\textit{end radius}'', tal campo não foi considerado devido à dificuldade de implementação, o que levou a diferenças entre o comprimento real total e das curvas e os calculados no programa.
		\paragraph{}
		Vale ressaltar que fora criada uma classe denominada ``Estatisticas'', a qual recebe informações da pista analisada durante toda a análise sintática (conhecida também pelo termo do inglês \textit{parsing}). Posteriormente à execução de todos os métodos do SAX, as seguintes informações estarão disponíveis: número de retas e curvas, comprimento total da pista, de todas as retas e de todas as curvas, somatório dos inversos dos ângulos de todas as curvas.
		\paragraph{}
		Com tais 

		\item \textbf{Metrificação}: citar todas as métricas desenvolvidas, seguidas de suas motivações, fórmulas e explicações detalhadas acima de seus raciocínios, além de apresentar as pistas escolhidas como base para os testes e as demais utilizadas ao longo do projeto, citando algumas informações sobre elas assim como o motivo de suas escolhas e os artigos de referência os quais foram buscados por apoio \cite{automaticgen} \cite{trackgen}.
	\end{itemize}
\section{Resultados}
	\begin{itemize}
		\item Apresentação dos resultados para cada métrica em cada pista da base (4 iniciais \cite{berniw} tendo dificuldade definida pela comunidade);
		\item Buscar uma relação entre a dificuldade definida pela comunidade e as métricas desenvolvidas, principalmente a busca por uma linearidade, um tipo de função monotônica que possa ser aplicada para uma fácil classificação;
		\item Caso possível, definir dentro da métrica níveis de dificuldade das pistas 
	\end{itemize}


\section{Conclusão}
	\begin{itemize}
		\item[$\diamond$] Citar os desafios encontrados ao longo do desenvolvimento;
		\item[$\diamond$] Resumo dos resultados;
		\item[$\diamond$] Debate sobre os resultados, com objetivo de verificar a utilidade da métrica;
		\item[$\diamond$] Análise do que pode ser melhorado no projeto: adição de uma rotina para adquirir e utilizar o campo ``end radius'' do arquivo .xml; implementação de uma variação para ser acoplado a um de piloto do SCRC, com o objetivo de analisar a pista dentro do jogo e metrifiá-la, possibilitando uma melhor seleção dos controles para cada prova, aprimoramento do método para aplicação em pistas do tipo \textit{``Dirt''}. 
	\end{itemize}


\bibliography{bibliografia}{}
\bibliographystyle{plain}
\end{document}