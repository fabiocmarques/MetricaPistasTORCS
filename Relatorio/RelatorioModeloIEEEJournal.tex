
%% bare_jrnl.tex
%% V1.4b
%% 2015/08/26
%% by Michael Shell
%% see http://www.michaelshell.org/
%% for current contact information.
%%
%% This is a skeleton file demonstrating the use of IEEEtran.cls
%% (requires IEEEtran.cls version 1.8b or later) with an IEEE
%% journal paper.
%%
%% Support sites:
%% http://www.michaelshell.org/tex/ieeetran/
%% http://www.ctan.org/pkg/ieeetran
%% and
%% http://www.ieee.org/

%%*************************************************************************
%% Legal Notice:
%% This code is offered as-is without any warranty either expressed or
%% implied; without even the implied warranty of MERCHANTABILITY or
%% FITNESS FOR A PARTICULAR PURPOSE! 
%% User assumes all risk.
%% In no event shall the IEEE or any contributor to this code be liable for
%% any damages or losses, including, but not limited to, incidental,
%% consequential, or any other damages, resulting from the use or misuse
%% of any information contained here.
%%
%% All comments are the opinions of their respective authors and are not
%% necessarily endorsed by the IEEE.
%%
%% This work is distributed under the LaTeX Project Public License (LPPL)
%% ( http://www.latex-project.org/ ) version 1.3, and may be freely used,
%% distributed and modified. A copy of the LPPL, version 1.3, is included
%% in the base LaTeX documentation of all distributions of LaTeX released
%% 2003/12/01 or later.
%% Retain all contribution notices and credits.
%% ** Modified files should be clearly indicated as such, including  **
%% ** renaming them and changing author support contact information. **
%%*************************************************************************


% *** Authors should verify (and, if needed, correct) their LaTeX system  ***
% *** with the testflow diagnostic prior to trusting their LaTeX platform ***
% *** with production work. The IEEE's font choices and paper sizes can   ***
% *** trigger bugs that do not appear when using other class files.       ***                          ***
% The testflow support page is at:
% http://www.michaelshell.org/tex/testflow/



\documentclass[journal]{IEEEtran}

\usepackage[utf8]{inputenc}
\usepackage[portuguese]{babel}
\usepackage{natbib}

% *** GRAPHICS RELATED PACKAGES ***
%
%\ifCLASSINFOpdf
  % \usepackage[pdftex]{graphicx}
  % declare the path(s) where your graphic files are
  % \graphicspath{{../pdf/}{../jpeg/}}
  % and their extensions so you won't have to specify these with
  % every instance of \includegraphics
  % \DeclareGraphicsExtensions{.pdf,.jpeg,.png}
%\else
 
%\fi



\begin{document}
%
% paper title
% Titles are generally capitalized except for words such as a, an, and, as,
% at, but, by, for, in, nor, of, on, or, the, to and up, which are usually
% not capitalized unless they are the first or last word of the title.
% Linebreaks \\ can be used within to get better formatting as desired.
% Do not put math or special symbols in the title.
\title{Criação e análise de métricas para as pistas do TORCS}
%
%
% author names and IEEE memberships
% note positions of commas and nonbreaking spaces ( ~ ) LaTeX will not break
% a structure at a ~ so this keeps an author's name from being broken across
% two lines.
% use \thanks{} to gain access to the first footnote area
% a separate \thanks must be used for each paragraph as LaTeX2e's \thanks
% was not built to handle multiple paragraphs
%

%\author{Fábio~Marques,~\IEEEmembership{Member,~IEEE,}}% <-this % stops a space
%\thanks{}% <-this % stops a space

\author{\IEEEauthorblockN{Fábio Marques}
\IEEEauthorblockA{Departamento de Ciência da Computação\\
Universidade de Brasília\\
Brasília, Distrito Federal 30332--0250\\
Email: fabio.cmarques@hotmail.com}}
%\and
%\IEEEauthorblockN{Homer Simpson}
%\IEEEauthorblockA{Twentieth Century Fox\\
%Springfield, USA\\
%Email: homer@thesimpsons.com}
%\and
%\IEEEauthorblockN{James Kirk\\ and Montgomery Scott}
%\IEEEauthorblockA{Starfleet Academy\\
%San Francisco, California 96678--2391\\
%Telephone: (800) 555--1212\\
%Fax: (888) 555--1212}}


% The paper headers
\markboth{Journal of \LaTeX\ Class Files,~Vol.~14, No.~8, August~2015}%
{Shell \MakeLowercase{\textit{et al.}}: Bare Demo of IEEEtran.cls for IEEE Journals}


% make the title area
\maketitle

% As a general rule, do not put math, special symbols or citations
% in the abstract or keywords.
\begin{abstract}
The abstract goes here.
\end{abstract}

% Note that keywords are not normally used for peerreview papers.
\begin{IEEEkeywords}
TORCS, Métricas, SCR, XML.
\end{IEEEkeywords}


% For peer review papers, you can put extra information on the cover
% page as needed:
% \ifCLASSOPTIONpeerreview
% \begin{center} \bfseries EDICS Category: 3-BBND \end{center}
% \fi
%
% For peerreview papers, this IEEEtran command inserts a page break and
% creates the second title. It will be ignored for other modes.
\IEEEpeerreviewmaketitle



\section{Introdução}

\IEEEPARstart{A}{qui} ficará a introdução, que há de ser escrita, contendo: toda a parte inicial do artigo, apresentação do problema e da proposta, citando também os méritos do projeto, sendo elas:  aplicação na área de geração procedural de conteúdo, de modo a criar pistas de acordo com uma dificuldade pré-definida por uma métrica;
% You must have at least 2 lines in the paragraph with the drop letter
% (should never be an issue)


% Colocam texto do lado direito da coluna
\hfill mds
 
\hfill August 26, 2015

\subsection{TORCS}
Apresentação do TORCS\cite{TORCS} e de algumas características do jogo;

% needed in second column of first page if using \IEEEpubid
%\IEEEpubidadjcol

\subsubsection{SCRC}
Apresentação do campeonato e colocá-lo motivador do projeto, assim como descrever uma futura implementação para esta versão do TORCS, seja para avaliação das pistas ou até mesmo de análise de desempenho dos pilotos;



\section{Metodologia}

\subsection{Pistas do TORCS}
  As pistas do TORCS\cite{TORCS} têm suas estruturas básicas descritas em arquivos de extensão ``.xml'', inclusive os segmentos que dão formas a elas, informações sobre as quais este trabalho teve maior foco. As diferenças de superfície não foram consideradas, devido a um aumento considerável da complexidade do projeto, que deveria também analisar e metrificar a influência dos diferentes coeficientes de atrito possíveis, o que tem potencial para ser uma área de pesquisa, já que não existem bibliografias sobre tal assunto; fazendo com que a análise feita fosse aplicada somente a pistas dos tipos ``\textit{oval}'' e ``\textit{road}'', definidos de acordo com a seção de pistas em \cite{berniw}. 

\subsection{Leitura de XML}
  Uma ferramenta na linguagem \textit{Java}\cite{Java} foi desenvolvida para a leitura dos arquivos ``.xml'' das pistas, fazendo uso da API popularmente conhecida como \textit{SAX}\cite{SAX}, disponível em \cite{meuGit}. As pistas e cada um dos segmentos foram modelados como objetos, sendo que este foi estendido para seus dois tipos básicos: curvas, com seus atributos de nome, arco, raio e para qual lado ela leva, tendo como referência de posição seu início (esquerda ou direita); e as retas, sendo compostas somente por seu nome e comprimento. As pistas são compostas somente por um objeto auxiliar, usado na aquisição das informações do arquivo, e uma lista ordenada de seus segmentos, de acordo com sua disposição no ``.xml''. Devido à heterogeneidade das definições das curvas, nas quais não existe um padrão de utilização do campo ``\textit{end radius}'', tal campo não foi considerado devido à dificuldade de implementação, o que levou a diferenças entre o comprimento real total e das curvas e os calculados no programa.
  \\
  %\paragraph{}
  A leitura de cada uma das informações é feita a partir da identifcação das marcações do arquivo. Como 9
  \\
  %\paragraph{}
  Vale ressaltar que fora criada uma classe denominada ``Estatisticas'', a qual recebe informações da pista analisada durante toda a análise sintática (conhecida também pelo termo do inglês \textit{parsing}). Posteriormente à execução de todos os métodos do SAX, as seguintes informações estarão disponíveis: número de retas e curvas, comprimento total da pista, de todas as retas e de todas as curvas, somatório dos inversos dos ângulos de todas as curvas. Com tais dados armazenados nos objetos já citados, é possível gerar todas as métricas desenvolvidas somente executando um método.




\section{Conclusão}
  Citar os desafios encontrados ao longo do desenvolvimento;
  \paragraph{Paragraph}
  Resumo dos resultados;
  \paragraph{Paragraph}
  Debate sobre os resultados, com objetivo de verificar a utilidade da métrica;
  \paragraph{Paragraph}
  Análise do que pode ser melhorado no projeto: adição de uma rotina para adquirir e utilizar o campo ``end radius'' do arquivo .xml; implementação de uma variação para ser acoplado a um de piloto do SCRC, com o objetivo de analisar a pista dentro do jogo e metrifiá-la, possibilitando uma melhor seleção dos controles para cada prova, aprimoramento do método para aplicação em pistas do tipo \textit{``Dirt''}.





% if have a single appendix:
%\appendix[Proof of the Zonklar Equations]
% or
%\appendix  % for no appendix heading
% do not use \section anymore after \appendix, only \section*
% is possibly needed

% use appendices with more than one appendix
% then use \section to start each appendix
% you must declare a \section before using any
% \subsection or using \label (\appendices by itself
% starts a section numbered zero.)
%


\appendices
\section{Tabelas com informações das pistas}
A tabela entra aqui.

% you can choose not to have a title for an appendix
% if you want by leaving the argument blank
\section{Tabelas com as métricas das pistas}
A outra tabela entra aqui


% use section* for acknowledgment
\section*{Reconhecimento}


The authors would like to thank...


% biography section
% 
% If you have an EPS/PDF photo (graphicx package needed) extra braces are
% needed around the contents of the optional argument to biography to prevent
% the LaTeX parser from getting confused when it sees the complicated
% \includegraphics command within an optional argument. (You could create
% your own custom macro containing the \includegraphics command to make things
% simpler here.)
%\begin{IEEEbiography}[{\includegraphics[width=1in,height=1.25in,clip,keepaspectratio]{mshell}}]{Michael Shell}
% or if you just want to reserve a space for a photo:

\begin{IEEEbiography}{Fábio Marques}
Aluno da UnB.
\end{IEEEbiography}

% if you will not have a photo at all:
\begin{IEEEbiographynophoto}{Fábio Marques}
Aluno da UnB também.
\end{IEEEbiographynophoto}

% insert where needed to balance the two columns on the last page with
% biographies
%\newpage



% You can push biographies down or up by placing
% a \vfill before or after them. The appropriate
% use of \vfill depends on what kind of text is
% on the last page and whether or not the columns
% are being equalized.

%\vfill

% Can be used to pull up biographies so that the bottom of the last one
% is flush with the other column.
%\enlargethispage{-5in}


\bibliography{bibliografia}{}
\bibliographystyle{plain}
% that's all folks
\end{document}


